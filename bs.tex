\title{How to win at Bullshit without even trying}
\author{
        James Baldwin-Brown \\
        Department of Biology\\
        University of Utah\\
        257 South 1400 East\\
        Salt Lake City, UT 84112, \underline{USA}
}
\date{\today}

\documentclass[12pt]{article}
\usepackage{tabu}
\usepackage{listings}

\begin{document}
\maketitle

\begin{abstract}
The traditional American card game of Bullshit is often discussed in terms of its
psychological strategy -- one must be a good judge of character to determine when
a player is lying or telling the truth about his or her cards. Here, I endeavor
to show that, even in this simple game of deception, the winning strategy relies
not on one's social ability to lie or tell the truth, but on the cold reality of the
cards. With a small amount of number theory, I produce a set of guidelines for winning
at the game without any guessing or any attempts at deception. The strategy relies
upon collecting as many cards as possible at the start of the game so that the player
has at least one card of each rank, then playing them sequentially while using the
vast knowledge granted by the player's card holdings to prevent others from winning.
\end{abstract}

\section{Introduction}
% Why games are thought to be psychological
A common phenomenon encountered when talking to card game players is that ``you don't
play the cards -- you play the person sitting across from you.'' By this, the player means
that a skilled player must be aware of social cues and psychological biases, and must
take into account the ``tells'' of other players to identify whether or not, say, a poker player is bluffing.
There is no evidence, however, that players are any better than chance at determining whether a fellow player
is bluffing. The ultimate example of this is the game ``Rock-Paper-Scissors'', in which
players must ascertain whether their opponent will play rock, paper, or scissors, and adjust
their play accordingly. Young children, upon encountering this game, quickly realize
that it is a game of luck, and the best move is to play as randomly as possible.
Yet, this game differs in no meaningful way from a game of poker in which one is determining
whether or not to call a bluff.

% Description of the game
A game that is often held up as a classic example of this ``psychological play'' is ``Bullshit'', a card game
in which players attempt to empty their hands of cards, in part by playing cards face-down and lying
about their face values. A description of the rules of the identical game ``Cheat'', from Wikipedia, follows below:

\begin{quote}
One pack of 52 cards is used for four or less players; five or more players should combine two 52-card packs. Shuffle the cards and deal them as evenly as possible among the players. No cards should be left. Some players may end up with one card more or less than other players.

The player who sits to the left of the dealer (clockwise) takes the first turn. They choose a card from his/her hand, lays it face down on the table and announces what rank it is (suit is irrelevant), which can be a lie. The 2nd player then adds a card to the pile of one rank higher than the previous card added, also face down, and announces what rank it is, which again can be a lie. Play continues like this until the game is over.

If any player thinks another player is lying, they can call the player out on it by announcing "Cheat" (or "Bluff," "I doubt it," etc.), and the card in question is revealed to all players. If the player was lying, they have to take the whole pile of cards into their hand. If the player was not lying, the caller must take the pile into their hand. Once the next player has placed a card, however, it is too late to call out any previous players.

The game ends if any player runs out of cards, at which point they win.
\end{quote}

% Why the game is thought to be social and psychological rather than numerical in its play
The game of ``Bullshit'' is thought to be a psychological game because, like
all card games, it contains an element of hidden information -- not all players
know the identity of all the cards. In addition, the game encourages players to
take advantage of the hidden information by including a rule in which players 
deliberately lie about information that they alone have access to. In truth,
however, the game is winnable without such trickery. While most card games
involve an element of chance, the element of chance that comes from this game
of lying-and-guessing can be entirely avoided by a sensible player.

\section{The order of play}

The central element that allows the player to win consistently without guessing
is the order of card play. Traditionally, the first player must play some
number of aces, the second player must play some number of deuces, and so on.
This pattern is consistently adhered to throughout the game such that, if the
cards are re-numbered such that the king is 0, the ace is 1, and the jack and
queen are 11 and 12, the following formula indicates the card to be played in
the next turn:

\begin{equation}
c_t = (c_0 + t) \bmod R
\end{equation}

Where $c_t$ is the rank to be played this turn, $c_0$ is the rank played on the
first turn (this turn is indexed 0), $t$ is the turn number, $\bmod$ is the
modulus (remainder) operator, and $R$ is the number of ranks in the deck.

Further, the player's next card to play is determined by the following equation:
\begin{equation}
c_p = (c_0 + pr) \bmod R
\end{equation}

Where $r$ is the number of rounds that the player has played and $p$ is the
number of players.

From this, we can reason that, in any set of $R$ rounds, the player will be
required to play each rank exactly once as long as the number of players $p$
does not share a common factor (apart from 1) with the number of ranks in the
deck $R$.  Conveniently, a typical card deck has 13 ranks, and 13 is a prime
number larger than the typical set of players for the game, so a player is
unlikely to encounter a scenario in which this strategy will not be successful.
As an aside, a 13-player (or 26-player) game would be nearly unwinnable, as
each player would only have access to one card rank that could be played
on-rank. For an example of the low-player-number scenario, in a four-player
game with a regulation 13-rank deck, the first player will encounter the
following cards in the first 14 turns:

\begin{lstlisting}
1 5 9 0 4 8 12 3 7 11 2 6 10 1
\end{lstlisting}

\begin{table}
    \begin{center}
        \begin{tabu}to 0.8\linewidth {XXX}
            Card Name & Numerical value & Alphabetic mapping\\
            \hline\\
            King & 0 & D\\
            Ace & 1 & A\\
            Deuce & 2 & K\\
            3 & 3 & H\\
            4 & 4 & E\\
            5 & 5 & B\\
            6 & 6 & L\\
            7 & 7 & I\\
            8 & 8 & F\\
            9 & 9 & C\\
            10 & 10 & M\\
            Jack & 11 & J\\
            Queen & 12 & G\\
        \end{tabu}
        \caption{Mapping to a new address space.}
        \label{table:map}
    \end{center}
\end{table}

For clarity, let us renumber the cards as in table \ref{table:map}. This allows
us to re-represent the above order as the following:

\begin{lstlisting}
A B C D E F G H I J K L M A
\end{lstlisting}

With the cards re-labeled, it is evident that each card will be played in a
specified sequence by each player. A corrolary of this is that, given the rules
of the game, assuming no other player wins in the meantime, a player that holds
one card of each rank is guaranteed to win in the next $R$ rounds of the game,
at which point they will have exhausted their entire hand.

More precisely, if a player is holding card ranks such that their next card to
play is the beginning of a sequential run that will exhaust their hand, they
will win the game in $X$ turns, where $X$ is the number of ranks in their hand.
For example, if player one in the above game is holding precisely the following
ranks in their hand:

\begin{lstlisting}
I J K L M A B C
\end{lstlisting}

and their next card to play is an I, then they can play all of their cards of a
given rank each turn, and their hand will look like the following turn-by-turn:

\begin{lstlisting}
J K L M A B C
K L M A B C
L M A B C
M A B C
A B C
B C
C
EMPTY -- GAME OVER
\end{lstlisting}

This holds true whenever the above conditions are met.

\section{The strategy}

Following on the understanding above of the order of the card playing, the
correct strategy is evident: the player must acquire the cards that allow the
construction of a run such as the one shown above.  The easiest way to acquire
enough contiguous ranks to ensure victory is to acquire as many cards as
possible in the early rounds of the game. To acquire cards rapidly, a player
should call ``bullshit'' every turn until they are on a winning run, and should
attempt to acquire more cards still by playing off-rank whenever it is their
turn.

Once the player is on a winning run, they should play all of their cards of a
given rank each turn until they are out of cards and have won. In order to
avoid disrupting the winning run, they should only call ``bullshit'' when the
player can be certain, based on their sizable hand and the currently discarded
cards, that another player's play is impossible.

Barring rare scenarios where a player is one or a few missing ranks away from
being in a winning run (in this case, attempting to ``bullshit'' through a few
rounds may be the preferred strategy, pending further analysis), the player
should never be in the position of needing to play an off-rank card and hope
they are not called on it.

\section{Conclusion}
Although players are routinely convinced that they can guess the intentions of
their fellow players, little evidence exists that people are capable of detecting
lies through social interaction. The game of ``Bullshit'' does not actually require
this ability, and a likely-winning game may be played without resorting to
guessing or hoping to avoid being called out at any time.

\end{document}
